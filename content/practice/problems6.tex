\documentclass[12pt]{paper} 
\usepackage{amsmath, amsfonts, amsthm} % Math packages
\usepackage{listings} % Code listings, with syntax highlighting
\usepackage[english]{babel} % English language hyphenation
\usepackage{graphicx} % Required for inserting images
\usepackage{booktabs} % Required for better horizontal rules in tables
\numberwithin{equation}{section} % Number equations within sections (i.e. 1.1, 1.2, 2.1, 2.2 instead of 1, 2, 3, 4)
\numberwithin{figure}{section} % Number figures within sections (i.e. 1.1, 1.2, 2.1, 2.2 instead of 1, 2, 3, 4)
\numberwithin{table}{section} % Number tables within sections (i.e. 1.1, 1.2, 2.1, 2.2 instead of 1, 2, 3, 4)

\setlength\parindent{0pt} % Removes all indentation from paragraphs

\usepackage{enumitem} % Required for list customisation
\setlist{noitemsep} % No spacing between list items

%----------------------------------------------------------------------------------------
%	DOCUMENT MARGINS
%----------------------------------------------------------------------------------------

\usepackage{geometry} % Required for adjusting page dimensions and margins

\geometry{
	paper=letterpaper, % Paper size, change to letterpaper for US letter size
	top=2.5cm, % Top margin
	bottom=3cm, % Bottom margin
	left=3cm, % Left margin
	right=3cm, % Right margin
	headheight=0.75cm, % Header height
	footskip=1.5cm, % Space from the bottom margin to the baseline of the footer
	headsep=0.75cm, % Space from the top margin to the baseline of the header
	%showframe, % Uncomment to show how the type block is set on the page
}

%----------------------------------------------------------------------------------------
%	FONTS
%----------------------------------------------------------------------------------------

\usepackage[utf8]{inputenc} % Required for inputting international characters
\usepackage[T1]{fontenc} % Use 8-bit encoding

\usepackage{fourier} % Use the Adobe Utopia font for the document

%----------------------------------------------------------------------------------------
%	HEADERS AND FOOTERS
%----------------------------------------------------------------------------------------

\usepackage{scrlayer-scrpage} % Required for customising headers and footers

\ohead*{} % Right header
\ihead*{} % Left header
\chead*{} % Centre header

\ofoot*{} % Right footer
\ifoot*{} % Left footer
\cfoot*{\pagemark} % Centre footer

%----------------------------------------------------------------------------------------
%	TITLE SECTION
%----------------------------------------------------------------------------------------

\title{	
Problem Set 6--STAT 7200
}

\author{} % Your name
\date{} % Today's date (\today) or a custom date

\begin{document}

\maketitle % Print the title

%----------------------------------------------------------------------------------------

\begin{enumerate}
	\item We discussed in class how $\hat{\mathbb{Y}} = P \mathbb{Y} = \mathbb{X}\hat{B}$ is the projection of $\mathbb{Y}$ on the column space $\mathcal{V}$ of $\mathbb{X}$:
	$$\mathcal{V} = \left\{\mathbb{X}A \mid A\mbox{ is a }(q+1)\times p\mbox{ matrix} \right\}.$$
	Use this fact to show that $\mathbb{X}\hat{B}$ is also the solution to the following least squares problem:
	$$\min_{V \in \mathcal{V}}\mathrm{tr}\left(\Omega(\mathbb{Y} - V)^T(\mathbb{Y} - V)\right),$$ 
	where $\Omega$ is an arbitrary $p\times p$ positive definite matrix.
	
	\vspace{1.5cm}\item For this question, we will use the \texttt{sheishu} dataset in the package \texttt{ACSWR}:
	\begin{verbatim}
	> library(ACSWR)
	> data(sheishu)
	\end{verbatim}
	Answer the following questions:
	\begin{enumerate}
		\item Find the least squares estimate $\hat{B}$ for the regression of (\texttt{Taste}, \texttt{Odor}) on the other eight covariates, and test for overall significance.
		\item Test the significance of (\texttt{Alcohol}, \texttt{Formyl\_nitrogen}) adjusted for the other covariates.
		\item Test the significance of (\texttt{Sake\_meter}, \texttt{Direct\_reducing\_sugar}, \texttt{Total\_sugar}) adjusted for the other covariates.
		\item Test the significance of (\texttt{pH}, \texttt{Acidity\_1}, \texttt{Acidity\_2}) adjusted for the other covariates.
		\item Using all subset selection, find the model with the lowest AIC.
	\end{enumerate}

	
\end{enumerate}

\end{document}
